% This LaTeX was auto-generated from MATLAB code.
% To make changes, update the MATLAB code and export to LaTeX again.

\documentclass{article}

\usepackage[utf8]{inputenc}
\usepackage[T1]{fontenc}
\usepackage{lmodern}
\usepackage{graphicx}
\usepackage{color}
\usepackage{hyperref}
\usepackage{amsmath}
\usepackage{amsfonts}
\usepackage{epstopdf}
\usepackage[table]{xcolor}
\usepackage{matlab}

\sloppy
\epstopdfsetup{outdir=./}
\graphicspath{ {./Final_images/} }

\matlabmultipletitles

\begin{document}

\matlabtitle{Signal and System}

\matlabtitle{Project (phaze1)}

\matlabtitle{Ali sadeghian 400101464}

\vspace{1em}

\begin{par}
\begin{center}
github address :https://github.com/BlackRanger4/Sys\_phaze1.git
\end{center}
\end{par}

\vspace{1em}

\matlabtitle{Q1}

\begin{par}
\begin{flushleft}
first part : $F^{-1} \left(1+H_{\mathrm{Hp}} \right)F$
\end{flushleft}
\end{par}

\begin{par}
\begin{center}
In this section, we are asked to make the original image more clear by adding a coefficient of
\end{center}
\end{par}

\begin{par}
\begin{center}
 high frequencies. High frequency in the image means extreme color changes that mostly occur 
\end{center}
\end{par}

\begin{par}
\begin{center}
at the edges of objects. By adding edges With the image itself, the clarity of the changes increases.
\end{center}
\end{par}


\vspace{1em}
\begin{par}
\begin{flushleft}
second part: $f+F^{-1} \left\lbrace 4\pi^2 \left(u^2 +v^2 \right)F\left(u,v\right)\right\rbrace$
\end{flushleft}
\end{par}

\begin{par}
\begin{center}
As argued in the previous section, by summing the high frequencies, the clarity will be increased, in this section,
\end{center}
\end{par}

\begin{par}
\begin{center}
 the high frequencies are amplified by a greater factor than the lower frequencies, so the image resolution is increased.            
\end{center}
\end{par}


\vspace{1em}
\begin{matlabcode}
clc
close all
clear all
\end{matlabcode}


\begin{par}
\begin{flushleft}
Part1:
\end{flushleft}
\end{par}

\begin{matlabcode}
% Read in images
I = imread('q1.jpg');
I_mirror = fliplr(I);
% Convert image to double precision
I = im2double(I);
% Compute the 2D Fourier transform of the image using fft2 function
F = fft2(I);
F_copy = F ;
[M, N , ~] = size(F);
% cut off f
D0 = 50;
% Create a high-pass filter
u = 0:(M-1);
v = 0:(N-1);
idx = find(u > M/2);
u(idx) = u(idx) - M;
idy = find(v > N/2);
v(idy) = v(idy) - N;
[V, U] = meshgrid(v, u);
D = sqrt(U.^2 + V.^2);
H = double(D > D0);
% Replicate the high-pass filter for each color channel
HH = zeros([M,N,3]);
HH(:,:,1) = H;
HH(:,:,2) = H;
HH(:,:,3) = H;
% Apply the high-pass filter to the Fourier transform of the image
k = 0.4 ;
G = (k*HH+1).* F;
GG = k*HH.* F;
% Compute the inverse 2D Fourier transform of the filtered spectrum using
% ifft2 function
g = real(ifft2(G));
gg =  real(ifft2(GG));
% Display original and filtered images side-by-side in a montage
montage({I_mirror,g,I_mirror,gg})
\end{matlabcode}
\begin{center}
\includegraphics[width=\maxwidth{84.61615654791771em}]{figure_0.eps}
\end{center}
\begin{matlabcode}
imshow(g)
\end{matlabcode}
\begin{center}
\includegraphics[width=\maxwidth{60.61214249874561em}]{figure_1.eps}
\end{center}


\begin{par}
\begin{flushleft}
FFT of picture
\end{flushleft}
\end{par}

\begin{matlabcode}
% Compute and display the log magnitude of each color channel of the Fourier 
% transform of the image
FR = abs(F_copy(:,:,1));
FB = abs(F_copy(:,:,2));
FG = abs(F_copy(:,:,3));

FR = log(FR);
FB = log(FB);
FG = log(FG);

imagesc(abs(FR))
colorbar
title("FFT Of R Chanal")
\end{matlabcode}
\begin{center}
\includegraphics[width=\maxwidth{60.61214249874561em}]{figure_2.eps}
\end{center}
\begin{matlabcode}
imagesc(abs(FB))
colorbar
title("FFT Of B Chanal")
\end{matlabcode}
\begin{center}
\includegraphics[width=\maxwidth{60.61214249874561em}]{figure_3.eps}
\end{center}
\begin{matlabcode}
imagesc(abs(FG))
colorbar
title("FFT Of G Chanal")
\end{matlabcode}
\begin{center}
\includegraphics[width=\maxwidth{60.61214249874561em}]{figure_4.eps}
\end{center}


\begin{par}
\begin{flushleft}
Part2:
\end{flushleft}
\end{par}

\begin{matlabcode}
% Compute the 2D Fourier transform of the image using fft2 function
F = fft2(I);
% Create a grid of spatial frequencies
[M, N,NN] = size(F);
u = 0:(M-1);
v = 0:(N-1);
idx = find(u > M/2);
u(idx) = u(idx) - M;
idy = find(v > N/2);
v(idy) = v(idy) - N;
[V, U] = meshgrid(v, u);
% Create a high-pass filter
H = 4*pi^2*(U.^2 + V.^2);
% Replicate the high-pass filter for each color channel
HH = zeros([M,N,3]);
HH(:,:,1) = H;
HH(:,:,2) = H;
HH(:,:,3) = H;
% Apply the high-pass filter to the Fourier transform of the image
G = HH.*F; 
% Compute the given function by adding a fraction of the inverse 2D Fourier 
% transform of the filtered spectrum to the original image
k = 1e-6;
Fpass = k*real(ifft2(G)); 
J = I + Fpass;

% Display original and filtered images side-by-side in a montage
montage({I_mirror, J,I_mirror, Fpass})
\end{matlabcode}
\begin{center}
\includegraphics[width=\maxwidth{84.61615654791771em}]{figure_5.eps}
\end{center}
\begin{matlabcode}
imshow(J)
\end{matlabcode}
\begin{center}
\includegraphics[width=\maxwidth{60.61214249874561em}]{figure_6.eps}
\end{center}


\matlabtitle{Q2}

\begin{par}
\begin{center}
In this section, we first categorize the size of the colors to find out what color each part of the photo is in, then we try to improve the brightness 
\end{center}
\end{par}

\begin{par}
\begin{center}
by applying a variable shift with the center of color focus (most histogram), then by applying High pass Gaussian filter, we increase the resolution
\end{center}
\end{par}

\begin{par}
\begin{center}
, then by detecting the sky and background, we subtract the original image from the background, then we add it with the sky, then we apply a filter
\end{center}
\end{par}

\begin{par}
\begin{center}
 to correct the contrast, and finally, by applying a histogram filter, we try We increase the brightness of the image.
\end{center}
\end{par}

\begin{matlabcode}
clc
clear all
close all
\end{matlabcode}


\begin{matlabcode}
% Read in images
II = imread('pic.jpg');
I = II;
figure
subplot(1,2,1)
imshow(I)
subplot(1,2,2)
imhist(I,256)
\end{matlabcode}
\begin{center}
\includegraphics[width=\maxwidth{56.196688409433015em}]{figure_7.eps}
\end{center}


\begin{matlabcode}
% Convert image to double precision
IP = double(II);
D = linspace(0,26,27)*10;
for i=1:26
    % Create a binary mask for each range of pixel values
    H(:,:,i) = IP>D(i) & IP<D(i+1);
end
\end{matlabcode}


\begin{matlabcode}
figure
% Display the binary masks for the first 26 ranges of pixel values
for i=1:6
    subplot(2,3,i)
    F(:,:,i) = H(:,:,i).*IP;
    imshow(uint8(F(:,:,i)))
end
\end{matlabcode}
\begin{center}
\includegraphics[width=\maxwidth{56.196688409433015em}]{figure_8.eps}
\end{center}
\begin{matlabcode}
figure
for i=7:12
    subplot(2,3,i-6)
    F(:,:,i) = H(:,:,i).*IP;
    imshow(uint8(F(:,:,i)))
end
\end{matlabcode}
\begin{center}
\includegraphics[width=\maxwidth{56.196688409433015em}]{figure_9.eps}
\end{center}
\begin{matlabcode}
figure
for i=13:18
    subplot(2,3,i-12)
    F(:,:,i) = H(:,:,i).*IP;
    imshow(uint8(F(:,:,i)))
end
\end{matlabcode}
\begin{center}
\includegraphics[width=\maxwidth{56.196688409433015em}]{figure_10.eps}
\end{center}
\begin{matlabcode}
figure
for i=19:24
    subplot(2,3,i-18)
    F(:,:,i) = H(:,:,i).*IP;
    imshow(uint8(F(:,:,i)))
end
\end{matlabcode}
\begin{center}
\includegraphics[width=\maxwidth{56.196688409433015em}]{figure_11.eps}
\end{center}
\begin{matlabcode}
figure
for i=25:26
    subplot(1,2,i-24)
    F(:,:,i) = H(:,:,i).*IP;
    imshow(uint8(F(:,:,i)))
end
\end{matlabcode}
\begin{center}
\includegraphics[width=\maxwidth{56.196688409433015em}]{figure_12.eps}
\end{center}


\begin{par}
\begin{flushleft}
shifting cololr value
\end{flushleft}
\end{par}

\begin{matlabcode}
Ipp = double(II);
[n , m] =size(I);
% Shifting color value using a piecewise cubic function with two segments 
D = 55;
A = 10;
for i=1:n
    for j=1:m
        temp = Ipp(i,j);
        if temp < D
            Error = (A/(D))*(temp -D) + (A/(D)^3)*(temp -D)^3 ;
            Ipp(i,j) = floor(Ipp(i,j)-Error)+1;
        else 
            Error = (A/((255-D)^4))*(temp - D)^4 + (A/(255-D))*(temp - D); 
            Ipp(i,j) = floor(Ipp(i,j)-Error);
        end
    end
end
% Display a montage of the original image and the modified image,
% as well as their histograms.
figure
figure
montage({II,uint8(Ipp)})
\end{matlabcode}
\begin{center}
\includegraphics[width=\maxwidth{107.73707977922732em}]{figure_13.eps}
\end{center}
\begin{matlabcode}
imhist(uint8(Ipp),256)
\end{matlabcode}
\begin{center}
\includegraphics[width=\maxwidth{107.73707977922732em}]{figure_14.eps}
\end{center}
\begin{matlabcode}
imhist(II,256)
\end{matlabcode}
\begin{center}
\includegraphics[width=\maxwidth{107.73707977922732em}]{figure_15.eps}
\end{center}
\begin{matlabcode}
imshow(uint8(Ipp))
\end{matlabcode}
\begin{center}
\includegraphics[width=\maxwidth{107.73707977922732em}]{figure_16.eps}
\end{center}


\begin{par}
\begin{flushleft}
image sharpening
\end{flushleft}
\end{par}

\begin{matlabcode}
% Image sharpening using a high-pass filter created by subtracting a 
% low-pass Gaussian filter from an average filter.
% Set filter size
filter_size = 128;
% Define the standard deviation of the Gaussian filter
sigma = 0.125;
% Create the low-pass Gaussian filter
h = fspecial('gaussian', filter_size, sigma);
h_high = fspecial('average', filter_size) - h;
% Apply the low-pass filter to the image
high_pass_filtered_image = imfilter(uint8(Ipp),h_high);
Fpass = uint8(Ipp) + high_pass_filtered_image *0.2;
montage({II, high_pass_filtered_image , uint8(Ipp) , uint8(Fpass)})
\end{matlabcode}
\begin{center}
\includegraphics[width=\maxwidth{107.73707977922732em}]{figure_17.eps}
\end{center}


\begin{par}
\begin{flushleft}
finding sky
\end{flushleft}
\end{par}

\begin{matlabcode}
% Finding sky by thresholding pixel values and removing non-sky points 
% using morphological opening.
% Create a copy of the image
Ip = double(Fpass);
% Get the size of the image
[n , m] =size(Ip);
% Loop through each pixel in the image
for i=1:n
    for j=1:m
         % Set pixel value to 0 if it is less than 208
        if Ip(i,j)<196
            Ip(i,j)=0;
        end
    end
end
% Display the modified image
imshow(uint8(Ip))
\end{matlabcode}
\begin{center}
\includegraphics[width=\maxwidth{107.73707977922732em}]{figure_18.eps}
\end{center}
\begin{matlabcode}
% Create a disk-shaped structuring element with radius
% 7 and 0 degrees of arc
se = strel('disk',5,0);
% Remove non-sky points from the image
Ipp = imopen(Ip,se);
Ipp = uint8(Ipp);
imshow(Ipp)
\end{matlabcode}
\begin{center}
\includegraphics[width=\maxwidth{107.73707977922732em}]{figure_19.eps}
\end{center}


\begin{par}
\begin{flushleft}
Remove the background from the original photo
\end{flushleft}
\end{par}

\begin{matlabcode}
% Create a disk-shaped structuring element with radius
% 9 and 6 degrees of arc
se = strel('disk',5,6);
% Perform morphological opening on the original image 
% using the structuring element to approximate the background
background = imopen(Fpass,se);
% Display the background approximation image
imshow(background)
\end{matlabcode}
\begin{center}
\includegraphics[width=\maxwidth{107.73707977922732em}]{figure_20.eps}
\end{center}
\begin{matlabcode}
% Subtract a fraction of the background approximation image 
% from the original image and multiply by a factor to adjust contrast
I1 = 1.5*(Fpass - 0.4*background);
% Display the result of contrast adjustment
imshow(I1)
\end{matlabcode}
\begin{center}
\includegraphics[width=\maxwidth{107.73707977922732em}]{figure_21.eps}
\end{center}
\begin{matlabcode}
% add sky
I2 = I1 + 0.1*Ipp;
imshow(I2)
\end{matlabcode}
\begin{center}
\includegraphics[width=\maxwidth{107.73707977922732em}]{figure_22.eps}
\end{center}


\begin{par}
\begin{flushleft}
applying contrast correction filter
\end{flushleft}
\end{par}

\begin{matlabcode}
% Set parameters for local Laplacian filtering
sigma = 0.9;
alpha = 1.05;
beta  = 3; 
numLevels = 128;
% Perform local Laplacian filtering on the enhanced image using the 
% specified parameters and add a fraction of the result of morphological
% opening to enhance edges and details further
I3 = locallapfilt(I2, sigma, alpha, beta, 'NumIntensityLevels', numLevels);
imshow(I3)
\end{matlabcode}
\begin{center}
\includegraphics[width=\maxwidth{107.73707977922732em}]{figure_23.eps}
\end{center}


\begin{matlabcode}
% Enhance contrast using adaptive histogram equalization with a custom
% clip limit
I4 = adapthisteq(I3,'NumTiles',[18 24],'ClipLimit', 0.003,'Distribution' ...
    ,'exponential','Alpha',0.35);
imshow(I4)
\end{matlabcode}
\begin{center}
\includegraphics[width=\maxwidth{107.73707977922732em}]{figure_24.eps}
\end{center}
\begin{matlabcode}
imshow(II)
\end{matlabcode}
\begin{center}
\includegraphics[width=\maxwidth{107.73707977922732em}]{figure_25.eps}
\end{center}


\matlabtitle{Q3}

\begin{par}
\begin{center}
In this section, by applying a low-pass Gaussian filter to Marlin's photo and a high-pass filter to 
\end{center}
\end{par}

\begin{par}
\begin{center}
Einstein's photo and adding the two, we try to create a hybrid photo.
\end{center}
\end{par}

\begin{matlabcode}
clc
clear all
close all
\end{matlabcode}


\begin{matlabcode}
% Read in images
Marilynl = imread('marilyn.jpg');
Einsteinl = imread('einstein.jpg');

% convert to gray picture
Einsteinl = rgb2gray(Einsteinl);
% Marilynl = rgb2gray(Marilynl);  Marilynl is gray

% Resize images to be of same size
Resize = 512;
Marilynl = imresize(Marilynl, [Resize Resize]);
Einsteinl = imresize(Einsteinl, [Resize Resize]);

% Convert images to double precision
Marilyn  = double(Marilynl) ;
Einstein = double(Einsteinl);

% Set filter size
filter_size = 64;
% Define the standard deviation of the Gaussian filter
sigma = 8;
% Create the low-pass Gaussian filter
h = fspecial('gaussian', filter_size, sigma);
% Apply the low-pass filter to the image
low_pass_filtered_image = imfilter(Marilyn, h);

% Create the high-pass filter by subtracting the low-pass filter from
% an all-pass filter
filter_size = 250;
sigma = 0.05;
h = fspecial('gaussian', filter_size, sigma);
h_high = fspecial('average', filter_size) - h;
% Apply the high-pass filter to the image
high_pass_filtered_image = imfilter(Einstein, h_high);
% Display original images side-by-side in a montage
montage({Einsteinl,Marilynl})
\end{matlabcode}
\begin{center}
\includegraphics[width=\maxwidth{84.61615654791771em}]{figure_26.eps}
\end{center}
\begin{matlabcode}
% Display filtered images side-by-side in a montage
montage({uint8(high_pass_filtered_image),uint8(low_pass_filtered_image)})
\end{matlabcode}
\begin{center}
\includegraphics[width=\maxwidth{84.61615654791771em}]{figure_27.eps}
\end{center}
\begin{matlabcode}
% Invert the high-pass_filtered_image color
white_high_pass_filtered_image = 255-high_pass_filtered_image;
% Display inverted and non-inverted high-pass filtered images 
% side-by-side in a montage
montage({uint8(high_pass_filtered_image),uint8( white_high_pass_filtered_image)})
\end{matlabcode}
\begin{center}
\includegraphics[width=\maxwidth{84.61615654791771em}]{figure_28.eps}
\end{center}
\begin{matlabcode}
% Combine the inverted high-pass filtered image and the low-pass 
% filtered image using weighted average 
% Interest of one half does not give the best result
combined_image=0.32*((white_high_pass_filtered_image))+0.68*((low_pass_filtered_image));
% Display combined image
montage({uint8(combined_image)})
\end{matlabcode}
\begin{center}
\includegraphics[width=\maxwidth{84.61615654791771em}]{figure_29.eps}
\end{center}

\end{document}
